\subsection{Implementation and Development}

This section details the technical realization of the \gls{NEST} ecosystem, integrating the firmware development, the simulation environment, and the service platform configuration. The implementation has been designed following a hybrid deployment approach, where physical and virtual devices coexist to ensure both the scalability and robustness of the poultry monitoring system.

To achieve a seamless integration, communication protocols and data structures have been unified, ensuring that both real hardware and simulated nodes interact identically with the central server. The following sections describe the technological pillars of this phase: the real-time system-based firmware for the physical device, the cyclic simulation engine in Python, and the data management architecture within the ThingsBoard platform.

\subsubsection{\acrshort{NEST} Device}

The firmware for the \gls{NEST} device has been developed using the \textbf{Arduino \gls{IDE}} framework. The software architecture leverages the \textbf{FreeRTOS} kernel integrated into the ESP32 core to implement a multitasking environment that ensures the system meets the real-time requirements of a smart farm.

To prevent race conditions when multiple tasks access the same hardware resources, the system implements \textbf{Semaphores (Mutexes)}. This is particularly critical for the \gls{SPI} bus, which is shared by the \gls{RFID} reader, and the \gls{MQTT} client used by both telemetry and attributes changes.

The system executes two main concurrent tasks to ensure efficient resource management:
\begin{itemize}
    \item \textbf{Telemetry Task:} Handles the periodic measurement of temperature, humidity, weight, and \gls{UID}. It transmits this data to the ThingsBoard platform every 10 seconds (configurable via remote attribute).
    \item \textbf{\gls{RFID} Task:} Operates asynchronously with a high sampling rate (250ms) to provide instant detection. This task allows for "Instant Publish" events, sending the identified farmer's \gls{UID} immediately upon detection without waiting for the telemetry cycle to be able to open the door.
\end{itemize}

The device implements a secure communication stack using \textbf{\gls{MQTT} over \gls{TLS} (Port 8883)} to interact with the ThingsBoard platform. This connection supports bidirectional traffic through a subscription to \textbf{Shared Attributes}, enabling "Commanding" from the server side as specified in the project requirements. 

The firmware utilizes the \texttt{mqttCallback} function to receive attributes updates. The integration of the \textbf{ArduinoJson} library is fundamental to this process, as it facilitates the parsing of complex \gls{JSON} payloads received from the server and the construction of structured telemetry messages for transmission.

The development of the \gls{NEST} device firmware relies on several specialized libraries to interface with the hardware components and manage data serialization.

\begin{table}[H]
    \centering
    \caption{External Libraries and Authorship}
    \label{tab:libraries}
    \begin{tabular}{p{0.25\textwidth}p{0.4\textwidth}p{0.25\textwidth}}
        \toprule
        \textbf{Library} & \textbf{Function} & \textbf{Author/Maintainer} \\ 
        \midrule
        DHT Sensor Library & Temp and Hum monitoring & Adafruit \\ 
        HX711 Arduino Library & Load cell data acquisition & Bogdan Necula, Andreas Molt \\ 
        MFRC522 & \gls{RFID} reader interfacing & Github Community \\ 
        ESP32Servo & \gls{PWM} control for SG90 motors & Kevin Harrington, John K. Bennett \\ 
        PubSubClient.h & \gls{MQTT} protocol implementation & Nick O'Leary \\ 
        WiFi.h and WiFiClientSecure.h & Wireless connectivity and \gls{TLS} encryption & Espressif Systems \\ 
        SPI.h & \gls{SPI} Bus management & Arduino / Espressif \\
        ArduinoJson & \gls{JSON} serialization for telemetry & Benoit Blanchon \\ 
        \bottomrule
    \end{tabular}
\end{table}

\subsubsection{\acrshort{NEST} Simulation}

To fulfill the requirement of a hybrid deployment where real and simulated worlds work together, a Python-based simulation environment has been developed. This tool allows the execution of multiple virtual nodes in parallel with the physical hardware, enabling scalability testing and verification of the system's logic across a larger network.

The simulation is built using the \texttt{paho-mqtt} library, which establishes a secure connection to the ThingsBoard platform via \textbf{\gls{MQTT} over \gls{TLS} (Port 8883)}. Each simulated instance runs a \gls{FSM} that replicates the life cycle of a nesting box and all its funtionalities. The defined states are:
\begin{itemize}
    \item \texttt{WAITING\_FOR\_HEN}: The default idle state with zero weight and no identified \gls{UID}.
    \item \texttt{HEN\_INSIDE}: Simulates a hen's presence by generating a random weight (2000g-3500g) and assigning a specific \gls{UID}.
    \item \texttt{EGGS\_DEPOSITED}: Represents the state where the hen has left (\gls{UID}: None), but the weight of the deposited eggs remains.
    \item \texttt{PERSON\_COLLECTING}: Simulates an authorized user collecting the eggs, identified by a unique \gls{UID}.
\end{itemize}

To evaluate the platform's performance with multiple devices, the system includes an automated deployment workflow. This consists of a \texttt{tokens.txt} provisioning file and a batch script (\texttt{run\_all.bat}) that launches multiple simulation instances simultaneously, each with its unique credentials and virtual identity.

\begin{table}[H]
    \centering
    \caption{Simulation Environment Components}
    \label{tab:sim_environment}
    \begin{tabular}{p{0.15\textwidth}p{0.8\textwidth}}
        \toprule
        \textbf{Component} & \textbf{Function} \\ 
        \midrule
        \texttt{nest\_sim.py} & Core script managing \gls{FSM} logic, sensors simulation, and secure \gls{MQTT} connectivity. \\ 
        \texttt{tokens.txt} & Provisioning list linking ThingsBoard devices access tokens with device names. \\ 
        \texttt{run\_all.bat} & Automation script for parallel execution of multiple virtual nodes. \\ 
        \bottomrule
    \end{tabular}
\end{table}

\subsubsection{ThingsBoard}

The centralized management of the \gls{NEST} ecosystem is performed through the ThingsBoard \gls{IoT} platform. This service acts as the core for data ingestion, device management, data visualization, and remote commanding.

\vspace{+10pt}
\textbf{Device Management}

To ensure a standardized configuration across the network, a specific \textbf{Device Profile} named \texttt{\gls{NEST} Device} has been created. This profile defines the common behavior, transport configurations, and alarm rules for all nodes in the coop. Currently, four operational devices have been provisioned under this profile:
\begin{itemize}
    \item \textbf{\gls{NEST} 1:} The primary physical node based on the ESP32 hardware.
    \item \textbf{\gls{NEST} 2, \gls{NEST} 3, and \gls{NEST} 4:} Simulated nodes running the Python-based \gls{FSM} to test hybrid deployment and scalability.
\end{itemize}

The platform utilizes \textbf{Shared Attributes} to enable remote commanding. These attributes are stored on the server and synchronized with the devices via \gls{MQTT}. As shown in \autoref{fig:shared_attributes}, the following key attributes are managed:
\begin{itemize}
    \item \texttt{door}: Controls the physical state of the nesting box (\texttt{open} or \texttt{closed}).
    \item \texttt{rgb}: Sets the color of the \gls{RGB} \gls{LED} (\texttt{Red, Green, Blue}).
    \item \texttt{period}: Defines the telemetry transmission interval in milliseconds.
    \item \texttt{state}: Current state of the \gls{FSM}.
    \item \texttt{eggs}: Estimated amount of eggs.
\end{itemize}

\begin{figure}[H]
    \centering
    \includegraphics[width=0.8\textwidth]{images/shared_attributes.png} 
    \caption{ThingsBoard Shared Attributes Configuration}
    \label{fig:shared_attributes}
\end{figure}

The platform utilizes \textbf{Server Attributes} to maintain the digital twin of each device. These attributes are modified on the server with the Rule Engine. As shown in \autoref{fig:server_attributes}, the following key attributes are managed:
\begin{itemize}
    \item \texttt{nest\_id}: A unique identifier for the specific physical location within the farm.
    \item \texttt{active}: Indicates whether the device is currently operational.
    \item \texttt{inactivityAlarmTime}: Time threshold for inactivity alarm generation.
    \item \texttt{lastActivityTime}: Timestamp of the last received telemetry data.
    \item \texttt{lastConnectTime}: Timestamp of the last successful connection to the server.
    \item \texttt{lastDisconnectTime}: Timestamp of the last disconnection from the server.
\end{itemize}

\begin{figure}[H]
    \centering
    \includegraphics[width=0.85\textwidth]{images/server_attributes.png} 
    \caption{ThingsBoard Server Attributes Configuration}
    \label{fig:server_attributes}
\end{figure}

The client can modify the \textbf{Client Attributes} to change configuration data on the device. These attributes can be read on the server side. As shown in \autoref{fig:client_attributes}, the following key attributes are managed:

\begin{itemize}
    \item \texttt{latitude / longitude}: Geographic latitude and longitude coordinates of the device.
    \item \texttt{maxTemp / minTemp}: Maximum and minimum temperature thresholds for alarm generation.
    \item \texttt{maxHum / minHum}: Maximum and minimum humidity thresholds for alarm generation.
    \item \texttt{avgWeight / minWeight}: Average and minimum weight thresholds to estimate egg presence.
    \item \texttt{validUID}: Authorized \gls{UID} for door access.
    \item \texttt{henUID}: \gls{UID} of the hen.
\end{itemize}

\begin{figure}[H]
    \centering
    \includegraphics[width=0.85\textwidth]{images/client_attributes.png} 
    \caption{ThingsBoard Client Attributes Configuration}
    \label{fig:client_attributes}
\end{figure}

The system performs continuous data ingestion from both physical and simulated sources. The telemetry stream provides real-time visibility into the farm's conditions. Each device pushes a \gls{JSON} payload containing:
\begin{itemize}
    \item \textbf{Environmental Data}: \texttt{temperature} and \texttt{humidity} levels.
    \item \textbf{Production Data}: \texttt{weight} readings from the load cells to detect egg presence.
    \item \textbf{Identification Data}: The \texttt{\gls{UID}} from the \gls{RFID} reader to identify the hen or authorized personnel currently at the nest.
    \item \textbf{Error Reporting}: An \texttt{error} key is included in the payload if the device detects any hardware malfunction.
\end{itemize}

\begin{figure}[H]
    \centering
    \includegraphics[width=0.9\textwidth]{images/telemetry_data.png}
    \caption{Real-Time Telemetry Data}
    \label{fig:telemetry_stream}
\end{figure}


\clearpage
\textbf{Rule Chain}

The centralized logic of the \gls{NEST} ecosystem is governed by a custom rule chain named \texttt{NEST Device}. This engine processes incoming data through a series of filtering, enrichment, and action nodes to automate the \gls{NEST}'s behavior and notify the user of critical events in real-time.

When a message reaches the server, a \textbf{Message Type Switch} node segregates traffic based on its nature. The system primarily handles the following interaction streams: 
\begin{itemize} 
    \item \textbf{Telemetry Processing:} Incoming telemetry—comprising \texttt{temperature}, \texttt{humidity}, \texttt{weight}, and \texttt{uid}—is validated by a \textbf{Check Telemetry} node. If the required keys are missing, a \texttt{MAJOR} alarm titled "Incorrect Message Syntax" is created. Valid data is persisted in the database via the \textbf{save data from sensor} node. 
    \item \textbf{Attribute Management:} The system processes \texttt{Post attributes} messages to update client-side configurations, such as geographic coordinates or sensor thresholds, using the \textbf{Save Client Attributes} node. 
\end{itemize}

The rule chain implements a comprehensive set of business logic to manage the nesting box lifecycle, environmental safety, and secure access:

\begin{itemize}
    \item \textbf{Device Initialization}: To ensure the system starts in a known state, the \textbf{Check Initialization} node detects \texttt{init} messages. Upon detection, the \textbf{Device Initialization} transformation node resets the \gls{FSM} to \texttt{Waiting for Hen}, clears the egg count, opens the door, and turns off the \gls{RGB} \gls{LED}.
    
    \item \textbf{Environmental Monitoring}: The system uses a "Fetch-and-Check" strategy. Metadata is enriched with thresholds (\texttt{maxTemp}, \texttt{minHum}, etc.) via the \textbf{TempRange} and \textbf{HumRange} nodes. Subsequently, \textbf{\gls{TBEL}} scripts in the \textbf{Check Temperature} and \textbf{Check Humidity} nodes evaluate if the real-time values fall outside these safe bounds, triggering or clearing \texttt{MAJOR} alarms accordingly.
    
    \item \textbf{Access Control and Interactive Feedback}: Door security is managed by the \textbf{UIDs} attribute node. If the incoming \texttt{uid} matches the \texttt{validUID} (farmer), the \textbf{Open Door} node toggles the \texttt{door} shared attribute and sets the \gls{LED} to \texttt{Green}. If the \texttt{uid} is unrecognized but not "None", the \textbf{Reject Door Use} node provides visual feedback by turning the \gls{LED} \texttt{Red}. The \textbf{Turn Off RGB} node ensures the \gls{LED} returns to an idle state after any interaction.
    
    \item \textbf{\gls{FSM} Integration}: A sophisticated \textbf{Finite State Machine (FSM)} tracks the biological and operational cycle: 
    \begin{itemize} 
        \item \textbf{Waiting for Hen / Hen Inside:} The \textbf{Check FSM State} node triggers a transition when the \texttt{henUID} is identified. 
        \item \textbf{Eggs Inside:} When the hen leaves and a residual weight is detected, the \textbf{Change FSM State} node calculates the number of eggs using the formula: $\text{round}(\text{weight} / \text{avgWeight})$. This triggers a \texttt{CRITICAL} \textbf{Eggs Inside} alarm.
        \item \textbf{Recolecting Eggs:} The state transitions to \texttt{Recolecting Eggs} when an authorized \texttt{validUID} is detected while eggs are present. Once the weight drops below \texttt{minWeight}, the system resets to the initial state.
    \end{itemize}

    \item \textbf{Telegram Notification System}: This implementation features a real-time alerting bridge. Whenever environmental alarms or \texttt{CRITICAL} production alerts (like egg detection) are triggered, a \textbf{Transformation Node} generates a formatted string. This payload is sent via a \textbf{\gls{REST} \gls{API} Call} node to the Telegram Bot \gls{API}, providing the farmer with immediate status updates and device identification directly on their mobile device.
    
    \item \textbf{Error Handling}: Reliability is reinforced by the \textbf{Check Error} node. If the hardware reports an internal failure through an \texttt{error} key, the \textbf{Hardware Error} node creates a \texttt{CRITICAL} alarm containing the specific error message and a timestamp, immediately notifying the user via Telegram to prevent data loss or animal distress.
\end{itemize}

\begin{figure}[H]
    \centering
    \includegraphics[height=0.9\linewidth, angle=270]{images/rule_chain.png}
    \caption{Rule Chain for \acrshort{NEST} Device}
    \label{fig:rule_chain}
\end{figure}


\vspace{+10pt}
\textbf{Dashboard}

The visualization layer is implemented through a multi-state interactive dashboard designed to provide both a global overview of the farm and detailed insights into individual nodes.

The primary dashboard state (see \autoref{fig:main_dashboard}) is designed for high-level monitoring. It integrates three main functional areas:
\begin{itemize}
    \item \textbf{Geographic Map:} Utilizing the \texttt{OpenStreet Map} widget, all four devices (\gls{NEST} 1-4) are geolocated based on their \texttt{latitude} and \texttt{longitude} attributes. The markers provide a quick visual reference of the sensors' physical distribution.
    \item \textbf{Entities Summary Table:} A real-time grid displaying the current \texttt{temperature}, \texttt{humidity}, \texttt{weight}, and \texttt{door status} for all devices in the network.
    \item \textbf{Global Alarms Feed:} A centralized list showing active alarms across the entire ecosystem.
\end{itemize}

\begin{figure}[H]
    \centering
    \includegraphics[width=0.95\textwidth]{images/main.png} 
    \caption{Main Dashboard View}
    \label{fig:main_dashboard}
\end{figure}

To enable drill-down capabilities, the dashboard implements a \textbf{state-based navigation} logic. By clicking on a specific device name in the table or its marker on the map, the dashboard transitions to a "Details" state (see \autoref{fig:details_view}). 

\begin{figure}[H]
    \centering
    \includegraphics[width=0.25\textwidth]{images/click.png} 
    \caption{Marker Information Popup}
    \label{fig:marker}
\end{figure}


This view focuses exclusively on the telemetry of the selected node:
\begin{itemize}
    \item \textbf{Indicator Cards:} Large, high-visibility widgets displaying the latest values for \texttt{Temperature}, \texttt{Humidity}, \texttt{Weight}, and current \texttt{Door} status.
    \item \textbf{Historical Trend Analysis:} A time-series line chart that allows the farmer to visualize the evolution of environmental variables and weight over time, which is essential for identifying patterns in egg-laying behavior.
\end{itemize}

\begin{figure}[H]
    \centering
    \includegraphics[width=0.95\textwidth]{images/details.png}
    \caption{\acrshort{NEST} Details Dashboard View}
    \label{fig:details_view}
\end{figure}

\subsubsection{Telegram}
As a way for visualization of the data and control the system, a Telegram bot named \texttt{NEST} has been created. In this bot, the farmer can see the data allocated in the telemetry fields, as well as the parameters configured in the attributes. Once the farmer has installed the bot, he will be receiving alarms from the \glspl{NEST}. The more common are the sensor's bounds ones.

As seen in the figure \ref{fig:boundaries_alarm}, when an alarm is created where the value of any sensor is out of the parameters configured a notification message is sent to telegram to inform about the sensor that has sent the alarm.
Once the value is stable again, another message is sent notifying that the alarm is unset.
\begin{figure}[H]
    \centering
    \includegraphics[width=0.5\textwidth]{images/Bounds_alarms.jpeg}
    \caption{Telegram Messages Boundaries Notification}
    \label{fig:boundaries_alarm}
\end{figure}

When an egg is laid, once the animal is out of the \gls{NEST}, a notification with the estimated number of eggs is sent to the bot, as well, when the eggs are recollected, another notification is sent, as demonstrated in figure \ref{fig:eggs_alarm}.

\begin{figure}[H]
    \centering
    \includegraphics[width=1\textwidth]{images/eggs_alarm.png}
    \caption{Eggs Telegram Messages}
    \label{fig:eggs_alarm}
\end{figure}

The telegram bots function with pre-programmed shortcuts to access some actions. For this project, five essential commands had been created, as display in figure \ref{fig:main_menu}:
\begin{itemize}
    \item \textbf{\texttt{/nest}}: It is the principal command of the bot. When this command is typed, a list of the available \glspl{NEST}, figure \ref{fig:nest_selection} is available to interact for the user.
    \begin{figure}[H]
        \centering
        \includegraphics[width=0.5\textwidth]{images/nest_selection_menu.jpeg}
        \caption{\acrshort{NEST} Selection Telegram Menu}
        \label{fig:nest_selection}
    \end{figure}
    \item \textbf{\texttt{/login}}: For some actions to use the system, a username and a password is required, with this command the user is able to log in the platform.
    \item \textbf{\texttt{/logout}}: If the user wants or needs to disconnect from the platform they can use this command.
    \item \textbf{\texttt{/status}}: Checks the status of the connection.
    \item \textbf{\texttt{/help}}: List all the commands that the bot can use with a little explanation of how to use it.
\end{itemize}
\begin{figure}[H]
    \centering
    \includegraphics[width=0.5\textwidth]{images/main_menu_bot.jpeg}
    \caption{Main Telegram Menu}
    \label{fig:main_menu}
\end{figure}

Once the \gls{NEST} is selected, another menu to interact with the system is displayed, figure \ref{fig:main_nest_menu}. The options in the menu are the following ones:
\begin{itemize}
    \item \textbf{Telemetry}: If logged, this option gives back the information of the telemetry in the \gls{NEST} selected.
    \item \textbf{Door Control}: Shows the current state of the door.
    \item \textbf{Temperature}: When selected, it shows the current temperature value and boundaries, it also shows how, if logged, change the boundaries of the temperature:
    \begin{itemize}
        \item \textbf{\texttt{/temperature max VALUE}}: With this command the user can change the maximum value of the temperature boundaries.
        \item \textbf{\texttt{/temperature min VALUE}}: With this command the user can change the minimum value of the temperature boundaries.
    \end{itemize}
    \item \textbf{Humidity}: Similar as the Temperature option, does the same things as the other option but using \texttt{/humidity} instead.
    \item \textbf{RGB}: Shows the current state of the \gls{LED}.
    \item \textbf{Eggs}: Shows the estimated quantity of eggs, that correspond with the selected animal. If logged, in this menu the user can change the type of animal that is using the \gls{NEST}, changing, in this way, the internal parameters of the system for the calculus of the egg count.
    \item \textbf{Location}: Shows the location, in coordinates, of the \gls{NEST} selected. If logged, the user can establish, by hand or using the \gls{GPS} location of the phone, the new coordinates of the \gls{NEST}.
    \item \textbf{Change \gls{NEST}}: A simple button for coming back to the previous menu for select the \gls{NEST}.
\end{itemize}
\begin{figure}[H]
    \centering
    \includegraphics[width=0.45\textwidth]{images/nest_menu.jpeg}
    \caption{Main \acrshort{NEST} Telegram Menu}
    \label{fig:main_nest_menu}
\end{figure}
